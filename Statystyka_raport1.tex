\documentclass{article}
\usepackage{titlesec}

\titleformat*{\section}{\LARGE\bfseries}
\titleformat*{\subsection}{\Large\bfseries}
\titleformat*{\subsubsection}{\large\bfseries}
\titleformat*{\paragraph}{\large\bfseries}
\titleformat*{\subparagraph}{\large\bfseries}
\usepackage[utf8]{inputenc}
\usepackage[T1]{fontenc}
\usepackage{lmodern}
\usepackage{polski}
\usepackage{mathtools,amsthm,amssymb,icomma,upgreek,xfrac,enumerate, bbm}
\usepackage{derivative}
\usepackage[hidelinks,breaklinks,pdfusetitle,pdfdisplaydoctitle]{hyperref}
\usepackage{geometry}
%\newgeometry{tmargin=0.5cm, bmargin=0.5cm, lmargin=0.5cm, rmargin=0.5cm}
\usepackage{multicol}
\usepackage{titling}
\usepackage{graphicx} 
\usepackage{url}
\setlength{\droptitle}{-1cm}
\mathtoolsset{showonlyrefs,mathic}
\title{Symulacje komputerowe raport 1}
\author{Adam Wrzesiński, Joanna Kołaczek}
\date{30.04.2022}
\newtheoremstyle{break}
{\topsep}{\topsep}%
{\normalfont}{}%
{\bfseries}{}%
{\newline}{}%
\theoremstyle{break}
\newtheorem{zadanie}{Zadanie} 
\newtheorem*{rozwiazanie}{Rozwiązanie}


%% KOMENDY:
\newcommand*{\e}{\mathrm{e}}
\newcommand{\hyline}[2]{%
	$#1$\> --\kern.5em #2 \\}


%% OPERATORY:
% `\diff` od „differential”, czyli odpowiednika słowa „różniczka” w języku
% angielskim.
\DeclareMathOperator{\diff}{d\!}
\newcommand{\indep}{\perp \!\!\! \perp}
\usepackage{amsmath}
\DeclareMathOperator{\EX}{\mathbb{E}}
\newcommand*{\E}{\mathrm{e}}

\usepackage{caption}
\usepackage[export]{adjustbox}
\usepackage{lipsum}
\usepackage{float}



\begin{document}
	\maketitle
	\tableofcontents
	\clearpage
\section{Wstęp}

\section{Poznane statystyki}
\subsection{Miary położenia}
\subsubsection{Średnia arytmetyczna}
$$\overline{x}=\frac{1}{n}\sum\limits_{i=1}^{n}x_i$$
\subsubsection{Średnia harmoniczna}
$$\overline{x}_h=\frac{n}{\sum\limits_{i=1}^{n}x_i}$$
\subsubsection{Średnia geometryczna}
$$\overline{x}_g=\sqrt{\prod\limits_{i=1}^{n}x_i}$$
\subsubsection{Średnia ucinana}
$$\overline{x}_u=\frac{1}{n-2k}\sum\limits_{i=k+1}^{n-k}x_i$$
\subsubsection{Średnia Winsorowska}
$$\overline{x}_w=\frac{1}{n}[(k+1)x_{k+1}\sum\limits_{i=k+2}^{n-k-1}x_i +(k+1)x_{n-k}]$$
\subsubsection{Kwantyle}
\paragraph{Mediana}
$$x_{med}=
\begin{cases}
	x, &  \text{gdy $n$ jest nieparzyste }\\
	\frac{1}{2}(x_{\frac{n}{2}}+x_{\frac{n}{2}+1}), &  \text{gdy $n$ jest parzyste}\\
\end{cases}$$
\paragraph{Kwartyle}
\begin{itemize}
	\item drugi kwartyl ($Q2$) to mediana
	\item pierwszy kwartyl ($Q1$) to mediana grupy obserwacji mniejszych od $Q2$
	\item drugi kwartyl ($Q3$) to mediana grupy obserwacji większych od $Q2$
\end{itemize}

\subsection{Miary rozproszenia}
\subsubsection{Rozstęp}
$$R=x_n-x_1$$
\subsubsection{IQR (rozstęp międzykwartylowy)}
$$IQR=Q3-Q1$$
\subsubsection{Wariancja z próby}
$$S^2=\frac{1}{n-1}\sum\limits_{i=1}^{n}(x_i - \overline{x})^2$$
\subsubsection{Odchylenie standardowe}
$$S=\sqrt{S^2}$$
\subsubsection{Współczynnik zmienności}
$$V=\frac{S}{\overline{x}} \quad (\cdot 100\%)$$

\subsection{Miary asymetrii}
\subsubsection{Współczynnik skośności (asymetrii)}
$$\alpha =\frac{n}{(n-1)(n-2)}\sum\limits_{i=1}^{n}(\frac{x_i-\overline{x}}{S})^3$$
\subsubsection{Inna postać współczynnika asymetrii}
$$\alpha_1 =\frac{\frac{1}{n}\sum\limits_{i=1}^{n}(x_i-\overline{x})^3}{(\frac{1}{n}\sum\limits_{i=1}^{n}(x_i-\overline{x})^2)^{\frac{3}{2}}}$$
\subsection{Miary spłaszczenia (koncentracji)}
\subsubsection{Kurtoza}
$$K_1 =\frac{\frac{1}{n}\sum\limits_{i=1}^{n}(x_i-\overline{x})^4}{(\frac{1}{n}\sum\limits_{i=1}^{n}(x_i-\overline{x})^2)^2}$$
$$K=\frac{n-1}{(n-2)(n-3)}((n+1)K_1-3(n+1))+3$$

\subsection{Miary położenia}
??

\section{Wizualizacja}
\subsection{Histogram}
\subsection{Boxplot}
\subsection{Dystrybuanta empiryczna}

\section{Podsumowanie}


\end{document}
